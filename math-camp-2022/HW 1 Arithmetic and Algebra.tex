\documentclass [12pt] {article}
\usepackage{fullpage}
\usepackage{setspace}
\usepackage{enumitem} 
\usepackage{amsmath}
\usepackage{amssymb}
\usepackage{array}
\newcolumntype{L}[1]{>{\raggedright\let\newline\\\arraybackslash\hspace{0pt}}m{#1}}
\newcolumntype{C}[1]{>{\centering\let\newline\\\arraybackslash\hspace{0pt}}m{#1}}
\newcolumntype{R}[1]{>{\raggedleft\let\newline\\\arraybackslash\hspace{0pt}}m{#1}}

\setlength\parindent{0cm}

\title{Math Camp 2022}
\author{Homework}
\date{August 12, 2022}
\begin{document}

\maketitle
\section{Arithmetic and Algebra Problems} 

Simplify as much as possible:

\begin{enumerate} 
	
	\item $\displaystyle\frac{1}{10} + \displaystyle\frac{1}{2} = \displaystyle\frac{3}{5}$
 
	
	\item $\displaystyle\frac{x^4}{x^3} = x$
	
	\item $b^3(a^3)^3 = (a^3b)^3$
    
    \item $\displaystyle\sum_{i=1}^{4}\bigg(\frac{1}{2}\bigg)^i 
    = \frac{1}{2} + \bigg(\frac{1}{2}\bigg)^2 +  \bigg(\frac{1}{2}\bigg)^3 + \bigg(\frac{1}{2}\bigg)^4 
    = \frac{8+4 + 2+1}{16}  =\frac{15}{16}$
    
    \item $\displaystyle\prod_{i=1}^{4}\bigg(\frac{1}{2}\bigg)^i = \frac{1}{2} * \bigg(\frac{1}{2}\bigg)^2 * \bigg(\frac{1}{2}\bigg)^3 * \bigg(\frac{1}{2}\bigg)^4 = {\displaystyle (\frac{1}{2})}  ^{\small(1+2+3+4)} =  \frac{1}{2^ {10}} $
	
\end{enumerate}

\vspace{12pt}
Solve for x: 

\begin{enumerate}
	\item $x-11 = 14, \;  x = 25$

	
	\item $x+15=30, \; x = 15$
	
	\item $12x=204 , \; x = 17$
	
	\item $\displaystyle\frac{x}{3}=9, \; x = 27 $

	\item $15x > 10 , \; x > {\displaystyle \frac{2}{3} }$
    
    \item $\displaystyle\frac{x}{-6} < 12 , \; x < -72$
\end{enumerate}

\clearpage 
Solve for $x$, $y$, and $z$ as applicable

\begin{enumerate}
    \item $x-7y = -11$ \\
    $5x + 27 = -18$

     \begin{align*} 
 x & = 7y - 11\\
5 (7y - 11)  +27 &= -18 \\
35 y &= 55 - 27 -18 = 10 \\ 
therefore, \\ 
y &= \frac{2}{7}\\
x &= -9
\end{align*}
    
    \item $-4x + 2z = 14$ \\
    $ y = x + z + 12 $ \\
    $ -2x -4z = 22 $

    \begin{align*}
        4x - 2z &= -14 \\ 
        -4x - 8z &=44 \\ 
        -10z &= 30 \\ 
        therefore, \\ 
        z &= -3\\
        x &= \frac{-14-6}{4} = -5\\
        y &= -3-5+12 = 4
    \end{align*}
    
\end{enumerate}
 
 
\newpage
\section{Set Theory}
\maketitle

Consider the following sets and find the following:\\

\begin{tabular}{@{}lr@{}}
    P = \{ UT GOV faculty members \} & $n(P) = 70$ \\
    T = \{ UT GOV faculty with tenure \} & $n(T) = 35$   \\
    U = \{ Unmarried UT GOV faculty members \} & $n(U) = 15$\\
M = \{ Married UT GOV faculty members \}& $n(M) = 55$\\
X = \{ Unmarried UT GOV faculty members with tenure \}& $n(X) = 10$\\
Y = \{ Married UT GOV faculty members with tenure \}& $n(Y ) = 25$\\
\end{tabular}

\bigskip Solve for the following: \\
 
\begin{enumerate}
    \item $n(T \cap M) = n(X) = 25 $
    \item $n(U \cup M) = n(P)=70$
    \item $n(T \cup M) = n(T +M -Y) = 35 + 55- 25 = 65$
    \item $n(\emptyset \cap T) = 0$
    \item $ U^C \cup U $ - is equivalent to $\mathbb{R}$
    \item $X \cup Y$ - is equivalent to $\mathbb{R}$
\end{enumerate}

\bigskip Evaluate the following expressions, simplifying your answers as much as possible. \\ 

\begin{enumerate}
    \item $ \displaystyle\sum_{i=1}^{6} (i^2 +3) = (1^2 +3) +(2^2 +3) +...+(6^2 +3) = 1+4 + 9+16+25+36 + 18 = 109 $
    \item $ {\displaystyle \frac{3! * 4!}{6!}}  = {\displaystyle \frac{1}{5} }$
\end{enumerate}


\newpage
\section{Functions, limit, and derivatives}
\maketitle

Solve the following for x.
\begin{enumerate}
    \item $ln(6e^4)=x \\ x = ln(6) + 4ln(e) = ln(6)+4$ 
    \item $\log_4 12 + \log_4 64 - \log_4 3 = x \\ x = log_4 (4*3)+ \log_4 (4^4) - \log_4 3 = 1+ \log_4 3 +4 -\log_4 3 = 5 $ 
    \item $\sqrt[x]{125} = 5 \\ x = 3 $
    \item $16^{3/2} = x \\ x = \sqrt[2]{16^3} = \sqrt[2]{16^3}$
    \item $8^{x-5}=64 \\ x = 7$
    \item $(x-2)(x-6) = 0 \\ x = 2 \; or \; x = 6$
    \item $x^2 +4x - 5 = 0 \\ (x+2)^2 - 9 = 0 \\ x+2 = \pm 3 \\ x = 1 \; or \; x = -5 $
\end{enumerate}

\bigskip
Find the following limits. Make sure to pay attention to whether or not a side has been specified (e.g., $\lim_{x \to \alpha^+} $ or $\lim_{x \to \alpha^-} $)
\begin{enumerate}
    \item $\displaystyle \lim_{x \to \infty} \frac{1}{x} +5 = 5 $
    \item $\lim_{x \to \infty} f(x) = 5$ and $\lim_{x \to \infty} g(x) = 2$, then $\lim_{x \to \infty} (f(x) +g(x))= 7$ 
    \item $ \displaystyle \lim_{x \to 2^+} { \frac{x^2+4x+4}{x^2-4}} \\ =\lim_{x \to 2^+} { \frac{(x+2)^2}{(x+2)(x-2)}} \\ =\lim_{x \to 2^+} { \frac{x+2}{x-2}} \\ =\lim_{x \to 2^+} (1 + { \frac{4}{x-2}}) \\ = -\infty $ when x approaches 2 from the right
    \item $ \displaystyle \lim_{x \to 2} { \frac{x^2-4x+4}{x^2-4}} \\ = \lim_{x \to 2} { \frac{(x-2)^2}{(x+2)(x-2)}} \\ = \lim_{x \to 2} \bigg( { \frac{x-2}{x+2}}\bigg) \\ = \lim_{x \to 2} \bigg(1 - {\displaystyle \frac{4}{x+2}} \bigg) \\ = 0$
    \item $ \displaystyle \lim_{x \to 9}{ \frac{x-9}{\sqrt{x}-3}} = \lim_{x \to 9}{ \frac{(\sqrt{x}-3)(\sqrt{x}+3)}{\sqrt{x}-3}}  =  \lim_{x \to 9}\bigg(\sqrt{x}+3\bigg) = 6$
    \item $ \lim_{x \to \infty} (16+d) = 16+d$
    \item $ \displaystyle \lim_{x \to 0} { \frac{-2x^4+32x^3 + 15x^2}{x^2} } \\=  \lim_{x \to 0} {\displaystyle \frac{x^2 (-2x^2+32x+15)}{x^2} } \\ =  \lim_{x \to 0}(-2x^2+32x+15) \\ = 15$
\end{enumerate}

\bigskip
Take the first derivative of the following with respect to x, and evaluate at x = 2.
\begin{enumerate}
    \item $(5x^3 - 2x^2 +1)\prime = 15x^2 - 4x = -30 $  at $x = 2$
    \item $(5e^x) \prime = 5e^x \approx 36.4$  at $x = 2$
    \item $ \displaystyle (ln(x) + ln (x^2)) \prime = \frac{1}{x} + (ln(x^2))\prime = \frac{1}{x}+ \frac{1}{x^2} * 2x = \frac{3}{x} = \frac{3}{2}$ at $x=2$
    \item $ (x^{\frac{5}{2}} + 6x^{\frac{4}{3}}) \prime = \frac{5}{2}x^{\frac{3}{2}} + 8x^{\frac{1}{3}} \approx 12.4 $  when $x=2$
    \item $ (6x^3 - 5e^x) \prime = 18x^2 -5e^x \approx 79.39$ when $x=2$
    \item $ \displaystyle x*ln(x) = 1*ln(x) + x* \frac{1}{x} = ln (x) +1$
    \item $ \displaystyle (\frac{e^x}{x^4})\prime = (e^x * x^{-4})\prime\\= (e^x)\prime (x^{-4}) +e^x*(x^{-4})\prime \\=  e^x(x^{-4}) -4x^{-5}e^x \\ ={ \frac{e^x}{x^4}(1-\frac{4}{x})}  \\ \approx 1.46 $ when $x=2$
    \item  $ \displaystyle (x^4 - \frac{6x^3}{ln(x)})\prime \\ = 4x^3 - { (\frac{6x^3}{ln(x)})} \prime \\ = 4x^3 - { \frac{18x^2 ln(x) -6x^3 *x ^ {-1}}{ln(x) ^2} } \\ = 4x^3 - { \frac{18x^2 ln(x) -6x^2}{ln(x) ^2} } \\ =  x^2 (4x - \frac{18}{ln(x)} + \frac{6}{ln(x) ^2})\approx 57.67  $ when $x=2$
    \item $ \displaystyle (\sqrt[3]{x} + \frac{1}{x^5}) \prime = (x^{\frac{1}{3}} + x^{-5}) \prime = \frac{1}{3}x^{-\frac{2}{3}} - 5x^ {-6} = {\frac{1}{3 \sqrt[3]{x^2}}}-\frac{5}{x^6} $
    
\end{enumerate}

\newpage
\section{derivatives and extreme}
\begin{enumerate}
    \item Take partial derivatives of the following with respect to x and z:
    \begin{enumerate}
        \item $x^2+z^2 + xz^3$ \\
            \begin{equation*}
            \begin{aligned}
            \frac{\partial (x^2+z^2 + xz^3)}{\partial x} &= 2x +z^3 \\
            \frac{\partial (x^2+z^2 + xz^3)}{\partial z} &= 2z + 3xz^2
            \end{aligned}
            \end{equation*}
    \item $ \displaystyle \frac{x}{z} - e^{xz}$
        \begin{equation*}
    \begin{aligned}
          \frac{\partial (\frac{x}{z} - e^{xz})}{\partial x}& = \frac{1}{z} -ze^xz\\
           \frac{\partial (\frac{x}{z} - e^{xz})}{\partial z} &= \frac{x}{z^2} -xe^xz
    \end{aligned}
        \end{equation*}
        \end{enumerate}
    \item Find extrema and state whether they are maxima or minima.
        \begin{enumerate}
            \item $x^3- 3x$ \\ 
            $ f'(x^3- 3x) = 3x^2 - 3 = 0, \; x = \pm 1 \\ 
            f''(x^3- 3x) = 6x \\
            f''(-1) = -6 <0, \; the \; maximum f(-1) = 2\\
            f''() = 6 >0, \; the \; minimum f(1) = -2$
            \item $x^5 - x$ \\ 
            $ f'(x^5 - x) = 5x^4 -1 = 0, \; x = \pm {\frac{1}{5}}^{\frac{1}{4}}\\ 
            f''(x^5 - x) = 20x^3 \\ 
            f''({\frac{1}{5}}^{\frac{1}{4}}) >0, \; minimum \; f({\frac{1}{5}}^{\frac{1}{4}}) = 30 \\
            f''(-2) = -160 <0, \; maxima f(-2) = $
                
        \end{enumerate}
\end{enumerate}
\newpage
\section{Integrals}
\begin{enumerate}
    \item Find the following indefinite integrals:
    \begin{enumerate}
        \item $ \displaystyle \int x^4 dx = \frac{1}{5}x^5 + c $
        \item $\displaystyle \int \frac{5}{x} = 5ln(|x|) + c $
        \item $\displaystyle \int \frac{5}{x^2} = \int 5x^{-2} = -\frac{5}{x} + c$
    \end{enumerate}
\bigskip
    \item Evaluate the following definite integral. Then, present your answer graphically and be pre
pared to explain how the graph relates to the equations you solved. 
\begin{enumerate}

    \item $ \displaystyle \int_{x=1}^{3} x^2 dx $ 
    \begin{equation*}
    \begin{aligned}
    F(x) &= \frac{1}{3}x^3 \\
    F(3) - F(1) &= 9 - \frac{1}{3} &= \frac{26}{3}
    \end{aligned}
    \end{equation*}
\end{enumerate}
\end{enumerate}

\newpage
\section{Matrices}

$A = 
\begin{bmatrix}
    4 &2&1\\ 6&-1&0\\ 3&1&2
\end{bmatrix} 
, \;
b = 
\begin{bmatrix}
    0&0&4\\ 1&2&3 \\ 10&3&2
\end{bmatrix}
,\;
C = \begin{bmatrix}
    1&2&3&4
\end{bmatrix}
,\;
D = \begin{bmatrix}
    3&2&1 \\ 2&3&1 \\ -4&-1&0 \\ -3&0&2
\end{bmatrix}
$ \\ 

\bigskip Given the above matrices, solve the following (or explain why a solution is not possible):
\bigskip
\begin{enumerate}
    \item $A+B =  \begin{bmatrix}  4&2&5\\7&1&3\\13&4&4 \end{bmatrix} $
    \item $A-B = \begin{bmatrix}
        4&2&-3\\5&-3&-3\\-7&-2&0
    \end{bmatrix}$
    \item $D-B $ is not conformable
   \item $D_{4*3}C_{1*4}$ is not conformable
    \item $C_{1*4}D_{4*3}= \begin{bmatrix}
       3+4-12-12 \\2+4-3 \\ 1+2+8\\
    \end{bmatrix} = \begin{bmatrix}
       17 \\3 \\ 11\\
    \end{bmatrix}$
    \item $A'B= \begin{bmatrix}
        4&6&3\\2&-1&1\\1&0&2
    \end{bmatrix} * 
    \begin{bmatrix}
    0&0&4\\ 1&2&3 \\ 10&3&2
\end{bmatrix} 
= \begin{bmatrix}
   0&0&12\\2&-2&3\\0&0&4 
\end{bmatrix}
$
    \item $AI_3=A$
    \item $A(3I_3)= \begin{bmatrix}
            4 &2&1\\ 6&-1&0\\ 3&1&2
\end{bmatrix}
    *
     \begin{bmatrix}
            3 &0&0\\ 0&3&0\\ 0&0&3
    \end{bmatrix} 
    = 
    \begin{bmatrix}
            12 &0&0\\ 0&-3&0\\ 0&0&6
    \end{bmatrix} 
    $

\end{enumerate}
\end{document}





